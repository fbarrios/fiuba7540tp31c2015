\documentclass[12pt,spanish,]{article}
\usepackage{lmodern}
\usepackage{amssymb,amsmath}
\usepackage{ifxetex,ifluatex}
\usepackage{fixltx2e} % provides \textsubscript
\ifnum 0\ifxetex 1\fi\ifluatex 1\fi=0 % if pdftex
  \usepackage[T1]{fontenc}
  \usepackage[utf8]{inputenc}
\else % if luatex or xelatex
  \ifxetex
    \usepackage{mathspec}
    \usepackage{xltxtra,xunicode}
  \else
    \usepackage{fontspec}
  \fi
  \defaultfontfeatures{Mapping=tex-text,Scale=MatchLowercase}
  \newcommand{\euro}{€}
\fi
% use upquote if available, for straight quotes in verbatim environments
\IfFileExists{upquote.sty}{\usepackage{upquote}}{}
% use microtype if available
\IfFileExists{microtype.sty}{\usepackage{microtype}}{}
\ifxetex
  \usepackage[setpagesize=false, % page size defined by xetex
              unicode=false, % unicode breaks when used with xetex
              xetex]{hyperref}
\else
  \usepackage[unicode=true]{hyperref}
\fi
\hypersetup{breaklinks=true,
            bookmarks=true,
            pdfauthor={},
            pdftitle={Algoritmos y Programación I (75.40) Trabajo práctico n.º 3},
            colorlinks=true,
            citecolor=blue,
            urlcolor=blue,
            linkcolor=black,
            pdfborder={0 0 0}}
\urlstyle{same}  % don't use monospace font for urls
% Make links footnotes instead of hotlinks:
\renewcommand{\href}[2]{#2\footnote{\url{#1}}}
\setlength{\parindent}{0pt}
\setlength{\parskip}{6pt plus 2pt minus 1pt}
\setlength{\emergencystretch}{3em}  % prevent overfull lines
\setcounter{secnumdepth}{0}
\ifxetex
  \usepackage{polyglossia}
  \setmainlanguage{}
\else
  \usepackage[spanish]{babel}
\fi

\title{Algoritmos y Programación I (75.40)\\Trabajo práctico n.º 3}
\date{Primer cuatrimeste 2015}
\usepackage[marginal,bottom,splitrule,stable]{footmisc}
\urlstyle{tt}

\begin{document}
\maketitle

\section{1. Introducción}\label{introducciuxf3n}

En la Universidad de Alfarería de la ciudad de Buenos Aires el querido
docente Toban Lajp ha desaparecido sin dejar rastros.

Aunque algunos afirman que ha ido de vacaciones a una pomposa isla del
Caribe habiendo apagado su celular, la idea que recorre los pasillos de
Paseo Colón es más sombría: los rumores cuentan que alguno de sus
allegados se ha encargado de él para siempre con el objetivo de
conseguir un aumento.

Una comitiva decidió que los sospechosos a investigar son:

\begin{itemize}
\itemsep1pt\parskip0pt\parsep0pt
\item
  Coronel D. Bárbara: al haber desaparecido Toban Lajp, es quien queda a
  cargo de su materia. Es ciertamente el más beneficiado.
\item
  Christian Grace: el extravagante joven millonario que tiene locas a
  las alumnas del curso de ``Logaritmos I'', y que se sospecha que hace
  tiempo está planeando maniobras para obtener protagonismo.
\item
  Haskell Martinez: siendo que se encuentra queriendo ejercer una
  carrera docente, esta desaparición puede catapultarlo hasta la cima.
\item
  Ing. Alan Información: la nueva personalidad extranjera que fue
  recibido con fiesta y panqueques, traida para jerarquizar la
  currícula. Sin embargo parecería querer escalar rápidamente hasta
  hacerse cargo del curso. ¿Vale?
\item
  Jesús: el gurú del grupo, y quien lo provee de vino y asados. Nadie
  desconfiaría de él, pero nunca se sabe\ldots{}
\item
  Lic. Pólez: el único sospechoso no allegado a la víctima. Se sospecha
  que la desaparición del Ing. Lajp y la debilitación de la materia
  puede tener intereses políticos.
\end{itemize}

Se decidió que se descubrirá el culpable a partir del juego
Clue\footnote{http://en.wikipedia.org/wiki/Cluedo\#Games}, pero nadie
cuenta con el famoso juego de mesa, ni tiene dinero para poder
compararlo en el corto plazo. Por lo tanto, se le pidió a un grupo muy
prestigioso de programadores que realicen el diseños y programación del
juego. Dicho grupo terminó de realizar todo el diseño, y de implementar
las interfaces y algunas partes del juego, pero al conocer a algunos
personajes de la Universidad de Alfarería se desat una discusión que
atentaba con una segunda desaparición misteriosa, por lo que será
necesario pedirle a alumnos con menor experiencia que terminen de
realizar el programa.

Por suerte, el código que se pudo recuperar está completamente
documentado, inclusive las partes faltantes, por lo que continuarlo no
debiera ser una tarea tan laboriosa.

\section{2. Consigna}\label{consigna}

Se pide implementar una variación del juego Clue.

Se entregan ya desarrollada e implementada parte de la funcionalidad
basica del programa. Esto incluye una interfaz gráfica, las funciones
principales del ciclo principal y una especificación de la interfaz de
comunicación entre los objetos del juego.

El alumno deberá completar las funciones faltantes, respetando las
especificaciones que se encuentran en el código fuente.

\newpage

\section{3. Criterios de aprobación}\label{criterios-de-aprobaciuxf3n}

A continuación se describen los criterios y lineamientos que deben
respetarse en el desarrollo del trabajo.

\subsection{3.1 Grupos}\label{grupos}

El trabajo práctico debe realizarse en grupo de dos personas.

\subsection{3.2 Informe}\label{informe}

El informe deberá consistir de las siguientes partes, según fueron
explicadas en clase:

\begin{itemize}
\item
  Diseño: diseño del programa y de las clases, atributos y métodos a
  desarrollar.
\item
  Implementación: Incluir aquí todo el código fuente utilizado,
  imprimiéndolo en tipo de letra \texttt{monoespaciado}, para facilitar
  su lectura.
\item
  Pruebas: Incluir todas las funciones desarrolladas que permitan
  verificar el correcto funcionamiento de las operaciones definidas para
  cada clase. No incluir capturas de pantalla.
\item
  Mantenimiento (opcional): posibles cambios a realizar en el trabajo,
  para mejorarlo.
\item
  También opcionalmente, toda explicación adicional que consideren
  necesaria, referencias utilizadas, dificultades encontradas y
  conclusiones.
\end{itemize}

El informe debe estar lo más completo posible, con presentación y
formato adecuados. Por ejemplo, este enunciado cumple con los
requerimientos de un informe bien presentado.

\subsection{3.3 Código}\label{cuxf3digo}

Además de satisfacer las especificaciones de la consigna, el código
entregado debe cumplir los siguientes requerimientos:

\begin{itemize}
\item
  El código debe ser claro y legible.
\item
  Todas las clases y funciones deben estar adecuadamente documentadas, y
  donde sea necesario el código debe estar acompañado de comentarios.
\item
  Además, claro, debe satisfacer la especificación de la interfaz.
\end{itemize}

\section{4. Entrega}\label{entrega}

La entrega del trabajo consiste en:

\begin{itemize}
\item
  El informe y código fuente impresos. Para el código fuente utilizar
  una tipografía \texttt{monoespacio}.
\item
  El informe digital, en formato \emph{PDF}.
\item
  Una versión digital de todos archivos \texttt{.py} de código,
  separados del informe. Al ser más de un archivo, se pide que estén
  comprimidos en un fichero \texttt{.zip}.
\end{itemize}

El informe impreso debe entregarse en clase. Los dos últimos (PDF y
código fuente) deben enviarse a la dirección electrónica
\texttt{tps.7540rw@gmail.com} con el asunto \emph{``TP3 -
\textless{}PADRÓN 1\textgreater{} - \textless{}PADRÓN
2\textgreater{}''}.

El plazo de entrega vence el \textbf{viernes 29 de mayo de 2015}.

\end{document}
